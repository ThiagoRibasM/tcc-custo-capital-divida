\section{Discussão}

Esta seção discute os resultados obtidos à luz da literatura empírica sobre determinantes do custo de capital de terceiros, comparando o desempenho do modelo com estudos anteriores e analisando a robustez das variáveis explicativas encontradas.

\subsection{Síntese dos Achados}

Os resultados oferecem evidências de que o custo de capital de terceiros das empresas brasileiras listadas no Novo Mercado é determinado por fatores de risco fundamentalistas. O modelo captura aproximadamente 27\% da variação no $K_d$, um poder explicativo consistente com estudos empíricos de estrutura de capital em mercados emergentes.

A significância estatística da alavancagem, tamanho e tangibilidade corrobora os pilares teóricos de \citet{modigliani1958} e \citet{myers1984}, enquanto o efeito do HHI de indexadores sugere que a estrutura de relacionamento bancário também desempenha papel relevante na precificação do risco de crédito corporativo.

\subsection{Comparação com a Literatura}

Para contextualizar os resultados, conduziu-se uma revisão de estudos empíricos sobre determinantes do custo de dívida. A seleção de artigos seguiu os seguintes critérios:

\begin{enumerate}
    \item \textbf{Metodologia:} Apenas estudos que utilizam Mínimos Quadrados Ordinários (OLS) ou variantes (e.g., efeitos fixos, erros robustos) foram considerados para comparação direta do $R^2$. Estudos que empregam métodos alternativos --- como Machine Learning, modelos de painel dinâmico (GMM), regressão quantílica ou redes neurais --- foram excluídos por não permitirem comparação direta do poder explicativo.
    
    \item \textbf{Variável Dependente:} Estudos que utilizam o custo de dívida corporativa como variável resposta, seja via spread de crédito, taxa de juros de empréstimos ou proxies contábeis.
    
    \item \textbf{Período e Escopo:} Preferência por estudos em mercados emergentes ou com amostras comparáveis (médias empresas, cross-section).
\end{enumerate}

A Tabela \ref{tab:literature} sintetiza os principais estudos comparáveis:

\begin{table}[H]
\centering
\small
\caption{Comparação com a Literatura Empírica}
\label{tab:literature}
\begin{tabular}{lccll}
\hline
\textbf{Estudo} & \textbf{N} & \textbf{$R^2$} & \textbf{Mercado} & \textbf{Variáveis Principais} \\
\hline
Este Estudo (2024) & 119 & 26.9\% & Brasil (B3) & Alav., Tam., Tang., HHI \\
Pittman \& Fortin (2004) & 371 & 24.0\% & EUA & Tam., Alav., Rating \\
Francis et al. (2005) & 3,143 & 31.0\% & EUA & Qual. Accruals, Tam., Alav. \\
Bharath et al. (2008) & 1,560 & 22.0\% & EUA & Spread, Alav., Rating \\
Goss \& Roberts (2011) & 3,996 & 28.0\% & EUA & ESG, Tam., Alav. \\
Kim et al. (2011) & 2,891 & 25.0\% & EUA & Controles Internos, Tam. \\
\hline
\end{tabular}
\vspace{0.3em}
\begin{flushleft}
\footnotesize Nota: Comparação restrita a estudos com metodologia OLS (ou variantes). Alav. = Alavancagem; Tam. = Tamanho; Tang. = Tangibilidade; HHI = Índice de Concentração.
\end{flushleft}
\end{table}

\subsubsection{Análise do Poder Explicativo}

O $R^2$ de 26.9\% obtido neste estudo situa-se dentro da faixa observada na literatura internacional (22\% a 31\%), sugerindo que o modelo captura proporção comparável da variância do custo de dívida. Estudos em mercados desenvolvidos, com maior disponibilidade de dados como ratings de crédito e spreads de mercado, tendem a apresentar $R^2$ marginalmente superior.

A Figura \ref{fig:comparison} ilustra visualmente o posicionamento deste estudo em relação à literatura.

\begin{figure}[H]
    \centering
    \includegraphics[width=\textwidth]{fig06_model_comparison.png}
    \caption{Comparação do Poder Explicativo com a Literatura}
    \label{fig:comparison}
\end{figure}

\subsubsection{Variáveis Robustas na Literatura}

A análise cruzada dos estudos revela que três variáveis aparecem de forma consistente como determinantes significativos do custo de dívida:

\begin{itemize}
    \item \textbf{Alavancagem:} Presente em 100\% dos estudos analisados. O sinal positivo (maior alavancagem $\rightarrow$ maior custo) é universalmente confirmado, refletindo a teoria de trade-off e o prêmio de risco de crédito.
    
    \item \textbf{Tamanho da Firma:} Presente em 100\% dos estudos. O sinal negativo (maior porte $\rightarrow$ menor custo) é consistente com a teoria de assimetria informacional --- empresas maiores são mais transparentes e têm melhor acesso a mercados de capitais.
    
    \item \textbf{Tangibilidade/Collateral:} Presente em aproximadamente 70\% dos estudos. Ativos físicos servem como garantia e reduzem o risco percebido pelos credores.
\end{itemize}

O presente estudo confirma esses três pilares fundamentais, sugerindo que os determinantes do custo de dívida operam de forma similar no mercado brasileiro em comparação com economias desenvolvidas.

\subsubsection{Contribuição Diferenciada: HHI de Indexadores}

Uma contribuição original deste estudo é a inclusão do Índice Herfindahl-Hirschman (HHI) de indexadores como proxy para heterogeneidade da dívida. Esta variável, derivada da granularidade das Notas Explicativas extraídas via LLM, captura uma dimensão raramente explorada na literatura tradicional.

O coeficiente negativo ($\beta = -1.85$, $p < 0.05$) sugere que empresas com estrutura de dívida mais concentrada em poucos indexadores tendem a obter melhores condições de financiamento. Esta evidência é consistente com a teoria de Relationship Banking --- relacionamentos de longo prazo com um número restrito de credores podem gerar vantagens informacionais e redução de custos de transação.

\subsection{Limitações}

Cabe ressaltar que a parcela não explicada pelo modelo (73\%) provavelmente reflete fatores não capturados neste estudo. A ausência de variáveis como o Rating de Crédito, indisponível para grande parcela da amostra, e de Garantias Específicas contratuais (covenants), que diferem da métrica contábil de tangibilidade, pode explicar parte da variância residual. Adicionalmente, o desenho cross-sectional impede a captura de variações temporais decorrentes de ciclos macroeconômicos, enquanto a não inclusão de dummies setoriais — decisão metodológica para preservar graus de liberdade — limita o controle por riscos idiossincráticos da indústria.

\subsection{Implicações Práticas}

Os resultados sugerem que gestores financeiros podem influenciar o custo de dívida através de políticas estratégicas, como a redução da alavancagem em períodos de maior percepção de risco e a manutenção de níveis adequados de ativos imobilizados para servirem como colateral. Além disso, a evidência sobre o HHI indica que a concentração estratégica de fontes de financiamento pode ser benéfica, fortalecendo relacionamentos bancários de longo prazo. Para investidores e credores, as variáveis identificadas confirmam-se como métricas fundamentais para a avaliação do risco de crédito de empresas listadas no Novo Mercado.
