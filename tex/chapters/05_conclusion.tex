\section{Considerações Finais}

O presente estudo investigou os determinantes do custo de capital de terceiros ($K_d$) das empresas listadas no segmento Novo Mercado da B3, utilizando uma abordagem inovadora que combina econometria tradicional com técnicas de Large Language Models (LLMs) para a extração de dados não estruturados.

Os resultados obtidos corroboraram a teoria clássica de estrutura de capital. Variáveis fundamentalistas como Alavancagem, Tamanho e Tangibilidade apresentaram significância estatística e sinais consistentes com as teorias de \textit{Trade-off} e Assimetria Informacional. Especificamente, observou-se que empresas maiores e com mais ativos tangíveis acessam crédito a custos menores, enquanto níveis elevados de endividamento penalizam o custo de captação devido ao prêmio de risco de crédito.

A principal contribuição deste trabalho reside na construção inédita do Índice Herfindahl-Hirschman (HHI) de indexadores de dívida, viabilizada pelo processamento das Notas Explicativas via LLMs. A relação negativa encontrada entre a concentração de indexadores e o custo da dívida oferece evidência empírica favorável à hipótese de \textit{Relationship Banking} no mercado brasileiro. Os dados sugerem que empresas que concentram suas fontes de financiamento tendem a desenvolver relacionamentos mais profundos com seus credores, reduzindo a assimetria informacional e obtendo melhores condições contratuais.

Em suma, este trabalho não apenas ratifica os determinantes clássicos do custo de dívida em um mercado emergente, mas também demonstra o potencial transformador das ferramentas de Inteligência Artificial Generativa para a pesquisa em finanças, permitindo a análise eficiente de variáveis qualitativas complexas anteriormente inacessíveis em larga escala.
