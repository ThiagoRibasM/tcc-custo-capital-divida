\section{Introdução}

A teoria de finanças corporativas moderna foi alicerçada sobre o pressuposto de mercados perfeitos \citep{modigliani1958}. Contudo, a presença de assimetria de informação eleva o custo de capital externo, criando a \textit{Pecking Order Theory} \citep{myers1984}. Quando o diferencial de custo torna-se proibitivo, as empresas enfrentam restrições financeiras.

\cite{fazzari1988} propuseram que a sensibilidade do investimento corporativo ao fluxo de caixa seria uma \textit{proxy} válida para o grau de restrição financeira. Entretanto, \cite{kaplan1997} contestaram essa visão, encontrando evidências opostas (empresas menos restritas com maior sensibilidade).

No contexto brasileiro, este debate é crucial devido ao mercado de crédito menos desenvolvido. Este trabalho inova ao utilizar Processamento de Linguagem Natural (LLMs) para identificar restrições financeiras diretamente dos relatórios de administração das empresas listadas na B3, confrontando essa classificação com a sensibilidade do investimento (métricas quantitativas).

O objetivo é analisar se empresas classificadas como "restritas" pelo modelo baseada em inteligência artificial apresentam comportamento de investimento distinto, contribuindo para o debate Fazzari-Kaplan-Zingales com uma nova abordagem metodológica.
