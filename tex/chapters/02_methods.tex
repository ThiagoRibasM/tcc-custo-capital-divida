\section{Metodologia}

\subsection{Processamento de Texto com LLMs}
Para mensurar a restrição financeira de forma qualitativa e em larga escala, adotou-se uma abordagem baseada em \textit{Large Language Models} (LLMs). O corpus textual compreende os Relatórios da Administração e Notas Explicativas das empresas listadas na B3, especificamente aquelas pertencentes ao Novo Mercado. Este segmento exige a adoção voluntária de práticas de governança corporativa adicionais às exigidas pela legislação. Suas principais características são a emissão exclusiva de ações ordinárias e a garantia de direitos igualitários aos acionistas minoritários em caso de alienação de controle (\textit{tag along}).

O processo de extração seguiu três etapas:
\begin{enumerate}
    \item \textbf{Recuperação de Documentos}: Coleta automatizada dos PDFs das Demonstrações Financeiras Padronizadas (DFP) e Notas Explicativas, seguida de conversão para formato textual (OCR/Parsing) para permitir processamento computacional.
    \item \textbf{Engenharia de Prompt}: Desenvolvimento de um \textit{System Prompt} especializado para atuária e finanças, instruindo o modelo a atuar como um especialista na extração de parâmetros de dívida corporativa. O prompt foi calibrado para identificar tabelas de endividamento e trechos narrativos complexos.
    \item \textbf{Extração de Informação com LLM}: Utilização do modelo \textbf{Google Gemini 1.5 Pro} (via API Vertex AI) para extrair dados estruturados de cada documento. Diferente de uma análise de sentimento genérica, o modelo foi configurado para retornar um objeto JSON contendo especificamente: custo da dívida ($K_d$), indexadores (ex: CDI, IPCA), moedas e prazos de vencimento. A saída estruturada elimina a necessidade de classificação manual binária, permitindo a construção direta de variáveis contínuas para o custo de capital.
\end{enumerate}

A Figura \ref{fig:llm_pipeline} sintetiza o fluxo completo deste processo, desde a recuperação dos documentos até a classificação final.

\begin{figure}[H]
    \centering
    \includegraphics[width=\textwidth]{fig01_llm_pipeline.pdf}
    \caption{Arquitetura do Pipeline de Dados e Ganho de Eficiência via LLM}
    \label{fig:llm_pipeline}
\end{figure}

\subsection{Seleção da Amostra e Filtros}
A amostra inicial é composta pelas 186 empresas listadas no segmento Novo Mercado da B3 no exercício de 2024. A partir deste universo, foram aplicados os seguintes filtros de saneamento:
\begin{itemize}
    \item Exclusão de empresas sem dados de endividamento estruturado nas Notas Explicativas, impossibilitando o cálculo do $K_d$ (59 empresas excluídas, restando 127).
    \item Exclusão de observação com dados financeiros insuficientes para cálculo das features (1 empresa excluída, restando 126).
    \item \textit{Winsorização} das variáveis contínuas a 1\% e 99\% para mitigar o efeito de \textit{outliers}.
    \item Remoção de observações influentes via Distância de Cook \citep{cook1977} ($D > 4/n$) (7 empresas excluídas, restando 119).
\end{itemize}
A amostra final utilizada na modelagem econométrica compreende \textbf{119 empresas} do Novo Mercado.

\subsection{Cálculo do Custo de Capital de Terceiros Ponderado ($K_d$)}

A variável dependente do modelo, o Custo de Capital de Terceiros ($K_d$), foi estimada a partir das informações extraídas das Notas Explicativas de cada empresa. O processo seguiu três etapas:

\textbf{Etapa 1 - Extração de Linhas de Financiamento.} Para cada empresa, o modelo LLM identificou nas Notas Explicativas os contratos de financiamento vigentes, extraindo:
\begin{itemize}
    \item Descrição do instrumento (ex: ``Capital de Giro'', ``BNDES'', ``Debêntures'')
    \item Indexador e \textit{spread} (ex: ``CDI + 4,91\% a.a.'', ``IPCA + 7,5\% a.a.'')
    \item Valor contábil consolidado em 31/12/2024 ($V_i$)
\end{itemize}

\textbf{Etapa 2 - Conversão para Taxa Nominal.} Os indexadores foram convertidos para taxas anualizadas utilizando as curvas de mercado de referência do período (CDI = 12,15\%, IPCA projetado = 4,5\%, SELIC = 12,25\%, TLP = 7,0\%). Para cada linha de financiamento $i$:

\begin{equation}
K_{d,i} = \text{Taxa\_Base}_i + \text{Spread}_i
\end{equation}

\textbf{Etapa 3 - Cálculo do Custo Ponderado.} O custo de capital de terceiros de cada empresa foi calculado como a média ponderada das taxas individuais, utilizando o valor contábil de cada financiamento como peso:

\begin{equation}
K_d = \frac{\sum_{i=1}^{n} K_{d,i} \times V_i}{\sum_{i=1}^{n} V_i}
\end{equation}

onde $n$ é o número de linhas de financiamento da empresa, $K_{d,i}$ é a taxa anualizada do financiamento $i$, e $V_i$ é o valor consolidado do financiamento $i$.

Esta abordagem difere de proxies simplificadas como ``Despesas Financeiras / Dívida Total'', por capturar a estrutura heterogênea da dívida corporativa e refletir as condições contratuais específicas de cada instrumento. A Figura \ref{fig:sample_summary} apresenta a caracterização da amostra final e a distribuição empírica do $K_d$ ponderado.

\begin{figure}[H]
    \centering
    \includegraphics[width=\textwidth, keepaspectratio]{fig02_sample_summary.pdf}
    \caption{Caracterização da Amostra e Distribuição do Custo de Capital de Terceiros}
    \label{fig:sample_summary}
\end{figure}

\subsection{Cálculo dos Indicadores Financeiros}

Para capturar a multidimensionalidade dos determinantes do custo de dívida, foram calculados indicadores financeiros organizados em seis blocos conceituais, conforme detalhado na Tabela \ref{tab:indicadores} e ilustrado na Figura \ref{fig:feature_mosaic}.

\begin{itemize}
    \item \textbf{Bloco 1 -- Alavancagem:} Indicadores de estrutura de capital que mensuram o nível de endividamento da empresa em relação a seus recursos próprios e ativos. Incluem: Dívida/Ativo Total, Dívida/Patrimônio Líquido (D/E), e Alavancagem Total ($D/(D+PL)$).
    
    \item \textbf{Bloco 2 -- Liquidez:} Indicadores que avaliam a capacidade de pagamento de curto prazo, incluindo Liquidez Corrente (AC/PC) e Liquidez Imediata (Caixa/PC).
    
    \item \textbf{Bloco 3 -- Rentabilidade:} Métricas de geração de valor e eficiência operacional: ROA, ROE, Margem EBITDA e Margem Líquida.
    
    \item \textbf{Bloco 4 -- Cobertura:} Indicadores que mensuram a capacidade de servir a dívida: Cobertura de Juros (EBITDA/Despesas Financeiras) e Dívida/EBITDA.
    
    \item \textbf{Bloco 5 -- Estrutura e Perfil:} Variáveis de controle que capturam características estruturais: Tamanho (log do Ativo Total), Tangibilidade (Imobilizado/Ativo) e proporção de dívida de curto vs. longo prazo.
    
    \item \textbf{Bloco 6 -- Heterogeneidade da Dívida:} Indicadores inspirados em \citet{fazzari1988} que capturam a diversificação das fontes de financiamento: Índice Herfindahl-Hirschman (IHH) por indexador e tipo de instrumento, e Índice de Diversificação (1 - IHH).
\end{itemize}

\begin{table}[H]
\centering
\small
\caption{Estatísticas Descritivas dos Indicadores Financeiros (N = 119)}
\label{tab:indicadores}
\begin{tabular}{llrrrrr}
\hline
\textbf{Bloco} & \textbf{Indicador} & \textbf{Média} & \textbf{DP} & \textbf{Mín.} & \textbf{Med.} & \textbf{Máx.} \\
\hline
Alavancagem & Alavancagem Total & 0.50 & 0.21 & 0.00 & 0.50 & 0.99 \\
Liquidez & Liquidez Corrente & 1.82 & 1.35 & 0.00 & 1.56 & 10.06 \\
Liquidez & Liquidez Imediata & 0.44 & 0.39 & 0.00 & 0.35 & 2.43 \\
Rentabilidade & ROA & 0.02 & 0.13 & -1.16 & 0.02 & 0.21 \\
Rentabilidade & Margem EBITDA & 0.18 & 0.29 & -1.71 & 0.20 & 1.99 \\
Estrutura & Tamanho (Log) & 16.11 & 1.71 & 12.18 & 15.75 & 22.18 \\
Estrutura & Tangibilidade & 0.24 & 0.22 & 0.00 & 0.18 & 0.72 \\
Heterogeneidade & IHH Indexador & 0.54 & 0.31 & 0.07 & 0.46 & 1.00 \\
Heterogeneidade & Índice Diversif. & 0.46 & 0.31 & 0.00 & 0.55 & 0.93 \\
\hline
\end{tabular}
\end{table}

A Figura \ref{fig:feature_mosaic} apresenta a metodologia de cálculo das features, a taxonomia dos blocos financeiros e as distribuições empíricas dos principais indicadores via \textit{violin plots}. Observa-se heterogeneidade significativa nas métricas de alavancagem e liquidez, enquanto os indicadores de rentabilidade apresentam maior concentração em torno da média.

\begin{figure}[H]
    \centering
    \includegraphics[width=\textwidth]{fig03_feature_mosaic.pdf}
    \caption{Criação de Features e Análise Exploratória}
    \label{fig:feature_mosaic}
\end{figure}

Para mitigar problemas de multicolinearidade, a seleção de variáveis para o modelo econométrico foi precedida por uma análise de correlação. A Tabela \ref{tab:correlation} apresenta a matriz de correlação de Pearson para as principais variáveis explicativas.

\begin{table}[H]
    \centering
    \caption{Matriz de Correlação de Pearson da Amostra Final}
    \label{tab:correlation}
    \resizebox{\textwidth}{!}{
    \begin{tabular}{lrrrrrrrr}
    \hline
     & Kd & Alavancagem & Liq.C. & ROA & M.EBITDA & Tamanho & Tangib. & Div. \\
    \hline
    Kd & 1.00 &  &  &  &  &  &  &  \\
    Alavancagem & 0.08 & 1.00 &  &  &  &  &  &  \\
    Liq.C. & -0.19 & -0.56 & 1.00 &  &  &  &  &  \\
    ROA & -0.22 & -0.16 & 0.16 & 1.00 &  &  &  &  \\
    M.EBITDA & -0.13 & -0.17 & 0.07 & 0.69 & 1.00 &  &  &  \\
    Tamanho & -0.36 & 0.03 & -0.17 & 0.17 & 0.16 & 1.00 &  &  \\
    Tangib. & -0.23 & 0.18 & -0.18 & 0.16 & 0.32 & 0.14 & 1.00 &  \\
    Div. & -0.01 & 0.11 & -0.21 & -0.06 & -0.11 & -0.00 & 0.01 & 1.00 \\
    \hline
    \end{tabular}
    }
    \vspace{0.3em}
    \begin{flushleft}
    \footnotesize Nota: Kd = Custo de Capital de Terceiros ponderado; Diversificação = Índice de Diversificação da Dívida (1 - HHI).
    \end{flushleft}
\end{table}

\subsection{Modelo Econométrico}

Para investigar os determinantes do custo de capital de terceiros, utiliza-se o modelo de Regressão Linear Múltipla estimado por Mínimos Quadrados Ordinários (OLS). A especificação geral do modelo é:

\begin{equation}
K_{d,i} = \beta_0 + \beta_1 X_{1,i} + \beta_2 X_{2,i} + \cdots + \beta_k X_{k,i} + \varepsilon_i
\label{eq:ols}
\end{equation}

\noindent onde $K_{d,i}$ é o custo de capital de terceiros ponderado da empresa $i$, $X_{j,i}$ representa a $j$-ésima variável explicativa, $\beta_j$ são os coeficientes a serem estimados, e $\varepsilon_i$ é o termo de erro aleatório com $E[\varepsilon_i] = 0$ e $Var[\varepsilon_i] = \sigma^2$.

\subsubsection{Seleção de Variáveis}

O processo de seleção de variáveis seguiu duas etapas:

\begin{enumerate}
    \item \textbf{Fator de Inflação da Variância (VIF):} Para mitigar multicolinearidade, variáveis com $VIF > 5$ foram progressivamente removidas. O VIF é calculado como:
    \begin{equation}
    VIF_j = \frac{1}{1 - R_j^2}
    \end{equation}
    onde $R_j^2$ é o coeficiente de determinação da regressão de $X_j$ sobre as demais variáveis explicativas.
    
    \item \textbf{Eliminação Backward:} Variáveis com p-valor $> 0.15$ foram sequencialmente removidas até que todas as remanescentes fossem estatisticamente significativas.
\end{enumerate}

\subsubsection{Tratamento de Outliers}

Observações influentes foram identificadas via Distância de Cook \citep{cook1977}, definida como:

\begin{equation}
D_i = \frac{(\hat{y}_j - \hat{y}_{j(i)})^2}{k \cdot MSE}
\end{equation}

\noindent onde $\hat{y}_{j(i)}$ é a predição para a observação $j$ quando a observação $i$ é excluída do ajuste, e $MSE$ é o erro quadrático médio. Observações com $D_i > 4/n$ foram removidas para garantir robustez dos estimadores.

\subsubsection{Correção de Heterocedasticidade}

A presença de heterocedasticidade foi avaliada pelo teste de Breusch-Pagan. Quando detectada ($p < 0.05$), os erros-padrão foram corrigidos utilizando o estimador robusto de variância \textbf{HC3} \citep{mackinnon1985}, que é consistente à heterocedasticidade:

\begin{equation}
\hat{V}_{HC3}(\hat{\beta}) = (X'X)^{-1} X' \text{diag}\left[\frac{\hat{u}_i^2}{(1-h_{ii})^2}\right] X (X'X)^{-1}
\end{equation}

\noindent onde $h_{ii}$ é o $i$-ésimo elemento diagonal da matriz chapéu $H = X(X'X)^{-1}X'$, e $\hat{u}_i$ são os resíduos OLS.

\subsubsection{Implementação Computacional}

A estimação foi realizada em Python utilizando a biblioteca \texttt{statsmodels} \citep{seabold2010}. O fluxo de execução inclui:

\begin{itemize}
    \item \texttt{sm.OLS(y, X).fit()}: Estimação OLS padrão
    \item \texttt{variance\_inflation\_factor()}: Cálculo do VIF
    \item \texttt{het\_breuschpagan()}: Teste de heterocedasticidade
    \item \texttt{.fit(cov\_type='HC3')}: Reestimação com erros robustos
\end{itemize}
